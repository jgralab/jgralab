\documentclass[a4paper]{article}
\usepackage[latin1]{inputenc}
\usepackage[T1]{fontenc}
\usepackage{mathptmx}
\usepackage[scaled=.9]{helvet}
\usepackage{courier}

\setlength{\parindent}{0pt}
\setlength{\parskip}{6pt}

\widowpenalty=10000
\clubpenalty=10000

\title{\LARGE XMI2TGSchema\\Manual}

\begin{document}

	\maketitle
	\vspace{-6pt}
	
	\section{Purpose}
	The tool \emph{XMI2TGSchema} transforms a TGraph schema specified by a grUML model in XMI-format to a corresponding TGraph schema definition in TG-format, readable by JGraLab. The possible deviations from grUML the tool is able to cope with are given in the descriptions of the \texttt{autoCorrect} parameter in section \ref{sec:Usage}. Up to now, the transformation has only been tested for XMI-files exported by Sparx System's \emph{Enterprise Architect}.
	
	\section{Prerequisites}
	As XMI2TGSchema is implemented as an XSL Stylesheet, an XSLT processor is needed for its execution. The processor must be capable of processing XSLT 2.0. The usage instruction below relate to \emph{Saxon-B}, available at \emph{http://www.saxonica.com}. The tested version of Saxon-B is 9.0.0.2J.
	
	\section{Usage} \label{sec:Usage}
	The command line instruction is:
	
	\texttt{java -jar saxon9.jar } \texttt{-s:}\emph{source}\texttt{ -xsl:}\emph{path}\texttt{/XMI2TGSchema.xsl} \\
	\mbox{[}\texttt{-o:}\emph{target}] \\
	\mbox{[}\texttt{appendEdgeIds=}(\texttt{yes}|\texttt{no})]\texttt{ }[\texttt{autoCorrect=}(\texttt{yes}|\texttt{no})] \\
	\mbox{[}\texttt{errorDetection=}(\texttt{yes}|\texttt{no})] \\
	\mbox{[}\texttt{extendedEdgeClassNames=}(\texttt{yes}|\texttt{no})] \\
	\mbox{[}\texttt{schemaName=}\emph{schemaName}]\texttt{ }[\texttt{tool=}\emph{tool}] \\
	\mbox{[}\texttt{uml=}(\texttt{yes}|\texttt{no})] \\

	with:
	\begin{itemize}
		\item \emph{source} -- The XMI-file to be transformed.
		\item \emph{path} -- The path to the XSL stylesheet \texttt{XMI2TGSchema.xsl}.
		\item \emph{target} -- The output file. If not given, the standard output device (usually the screen) is used.
	\end{itemize}
	\begin{itemize}
		\item \emph{appendEdgeIds} -- If set to \texttt{yes}, ids of associations are added to the names of corresponding EdgeClasses. This helps to avoid duplicate names, but makes code harder to understand. Default is \texttt{no}.
		\item \emph{autoCorrect} -- If set to \texttt{yes}, the following corrections are made:
		\begin{itemize}
			\item If the first character of Vertex Class and Edge Class names is lower case, it is set to upper case.
			\item Identifiers conflicting with reserved words are modified by prepending an apostrophe (').
			\item If an Edge Class has no name, it is created by using role names or the names of the incident Vertex Classes. 
		\end{itemize}
		It is recommended to set this parameter to \texttt{yes}. Default is \texttt{yes}.
		\item \emph{errorDetection} -- If set to \texttt{yes},
		\begin{itemize}
			\item Vertex Classes without names are detected and an error is thrown.
			\item Vertex Classes with duplicate names are detected and an error is thrown.
			\item a Schema which is not self-contained is detected and an error is thrown\footnote{Most probably an association in the XMI-file is to be transformed to an Edge Class, but one of the associated Classes is missing.}.
		\end{itemize}
		It is recommended to set this parameter to \texttt{yes}. Default is \texttt{yes}.
		\item \emph{extendedEdgeClassNames} --  If set to \texttt{yes} names of EdgeClasses correspond to the extended form \emph{FromRolename}\texttt{LinksTo}\emph{ToRolename}, else names correspond to \texttt{LinksTo}\emph{ToRolename}. Default is \texttt{no}.
		\item \emph{schemaName} -- The name of the created Schema. Must correspond to the UML package in the modeling tool which contains the schema. If the package does not contain a class with stereotype \texttt{<<GraphClass>>}, the name of the created GraphClass is \texttt{DefaultGraphClass}.
		\item \emph{tool} -- The tool used to create the XMI file. As yet, the only possible value is \texttt{ea}. Omission of this parameter probably results in corrupted TG files (e.g.\ due to missing attribute types).
		\item \emph{uml} -- If set to \texttt{yes}, the grUML equivalents of the following UML constructs are created:
		\begin{itemize}
			\item associations with \emph{derived union} association end => \emph{abstract} edge classes
			\item associations with subsetting association end => EdgeClass as specialization of edge class corresponding to association with subsetted association end.
		\end{itemize}
		Default is \texttt{no}.
	\end{itemize}
	
	\section{Notes}
	\subsection{Conversion of associations}
	As it is not possible to directly specify package containment for associations in Enterprise Architect, the containment in the exported XMI is derived from the classes connected by the association.

	The following observations were made:
	\begin{itemize}
		\item If both classes reside in the same package, the association is also considered to be in this package.
		\item If one class resides in a package which is ``above'' the other class's package in the package hierarchy, the association is considered to be in the package on the higher hierarchy level.
		\item If the classes reside in different packages on the same hierarchy level, package containment for the concerned association seems to be random.
	\end{itemize}
\end{document}