\newcommand{\makeisttitle}[2] % 1st argument author; 2nd argument 
{\begin{center}
  \begin{tabular*}{\textwidth}{ll@{\extracolsep{\fill}}r}
  \multirow{2}{20mm}[3mm]{\includegraphics[width=22mm]{istc.pdf}}
   & \begin{large}\textit{\textsf{Institut für}}\end{large} & \multirow{2}{7cm}[2mm]{\includegraphics[width=7cm]{ulogoc.pdf}}\\

   & \begin{large}\textit{\textsf{Softwaretechnik}}\end{large} & ~\\

 ~ & ~ & \textsf{Fachbereich 4: Informatik} \hspace{8.8mm} \\
  \end{tabular*}
\end{center}


%\maketitle
\begin{center}

\vspace{3cm}
{\Large #1}\\
\vspace{0,5cm}
#2

\end{center}}

% special commands
\newcommand{\rselem}[1]{\textsf{#1}}

% layout
\geometry{verbose, a4paper, tmargin=3.5cm, bmargin=3cm, lmargin=3cm, rmargin=2cm, headheight=2cm, headsep=1.5cm, footskip=1.5cm}
\setlength\parskip{\medskipamount}
\setlength\parindent{0pt}
\onehalfspacing
\setcounter{tocdepth}{2}


% headings
\pagestyle{fancy}
\renewcommand{\headrulewidth}{0pt} 
\renewcommand{\footrulewidth}{0pt}

\newcommand{\setistheader}[1]{
\chead{#1}
\rhead{}
\lhead{}
}


%Change fontsize of /section /subsection and /subsubsection


\lstdefinelanguage{greql2}{
 morekeywords={using, from, end, with, report, in, forall, exists!, exists, as,
 list, set, bag, rec, path, pathsystem, thisVertex, thisEdge, thisGraph,
 reportBag, not, reportSet, tup, and, andThem, or, orElse, xor, V, E, eSubgraph, vSubgraph,
 let, where, store },
 sensitive=true,
 morecomment=[l]{//},
 morecomment=[n]{/*}{*/},
 morestring=[b]",
}

\lstdefinelanguage{greql1}{
 morekeywords={USING, FROM, END, WITH, REPORT, IN, FORALL, EXISTS, AS, LIST, SET, BAG, NOT, TUP, AND, OR, XOR, V, E, LET, WHERE}
 sensitive=true,
 morecomment=[l]{//},
 morecomment=[n]{/*}{*/},
 morestring=[b]",
}



\newcommand{\fnspccorr}{\hspace{2.6pt}}

\newcommand{\Footnote}[1]{\footnote{\fnspccorr#1}}

\newcommand{\Fntext}[1]{\footnotetext{\fnspccorr#1}}

\newcommand{\standardfarbe}{black}
\newcommand{\beispielfarbe}{darkblue}

\newcommand{\ueberabsatz}[1]{

~

{\itshape #1}\\}

\newcommand{\sindentalt}[1]{%
\setlength{\remainwidth}{\linewidth}%
\addtolength{\remainwidth}{-\indentvalue}%
%\addtolength{\remainwidth}{-2pt}%
\newline\noindent\hspace*{\indentvalue}%
\parbox[b]{\remainwidth}{#1}%
}


\clubpenalty=10000 %Einzelne Zeilen neuer Absäze an
\widowpenalty=10000 %auf einer Seite oben oder unten
\displaywidowpenalty=10000 % ausschließen

\lstset{
	basicstyle=\ttfamily\scriptsize,
	stringstyle=\ttfamily,
	showstringspaces=false,
	numbers=left,
	numberstyle=\tiny
}


\newcommand{\cnf}[1]{\texttt{#1}} %ClassNameFormat
\newcommand{\gnf}[1]{\textsc{#1}} %GuproNameFormat
\newcommand{\frage}[1]{\textcolor{red}{\textbf{#1}}}
\newcommand{\todo}[1]{\textcolor{red}{\textbf{#1}}}
\newcommand{\greql}[1]{\texttt{#1}}

\newcommand{\bspindent}{30pt}
\newcommand{\lnsmallskip}{5pt}


\setcounter{totalnumber}{1}
%\setcounter{secnumdepth}{3}


\newcommand{\edgeto}[2] %Kante #2 von #1 nach #3
  {\ensuremath{ \stackrel{\cnf{#1}}{\longrightarrow} \cnf{#2}}}
\newcommand{\edgefrom}[2] %Kante #2 von #1 nach #3
  {\ensuremath{ \stackrel{\cnf{#1}}{\longleftarrow} \cnf{#2}}}

\newcommand{\edge}[3] %Kante #2 von #1 nach #3
  {\ensuremath{#1 \stackrel{#2}{\rightarrow} #3}}

\newcommand{\edgeset}[1] %Menge der Kante mit der Beschriftung #1
  {\edge{}{#1}{}}

\newcommand{\Path}[3] %Pfad mit der Beschriftung #2 von #1 nach #3
  {\ensuremath{#1 \stackrel{#2}{\shortrightarrow\shortrightarrow} #3}}

\newcommand{\pathset}[1] %Menge der Pfade mit der Beschriftung #1
  {\Path{}{#1}{}}

\newcommand{\CompletePath}[4]%#1: Pfad #2: Knoten #3: Kanten #4: L�ge
  {\ensuremath{#1 = ({#2}_0, {#3}_1, {#2}_1, {#3}_2, \ldots, {#2}_{#4-1}, {#3}_{#4}, {#2}_{#4})}}

\newcommand{\insertop}[2]%#1: Ziel #2: Quelle
  {\ensuremath{#1 \ll #2}}